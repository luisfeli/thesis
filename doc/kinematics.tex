\documentclass[a4paper]{report}

\begin{document}

\section{Kinematics}
Kinematics is the science that studies the geometry of motion. It is restricted to a purely geometrical description of motion by means of position, orientation and their time derivatives. 

In general robots of any kind can be modeled as a set of articulated rigid bodies

Kinematics helps to answer the following two questions:
Given a set of values for the different joint parameters where would the end effector (hand, foot, etc.) be located? This is know as forward kinematics.
If we wanted to position the end effector in a particular position for instance we wanted to reach the door's knob or we wanted to place our foot in the right spot not fall, How should the joint parameters be adjusted? This is known as inverse kinematics.

For the present work the second problem is more interesting, but it is also more complicated as multiple solutions may exist for the same problem <reference>. However this is something we solve every day, we never stop to think if we should bent our knee 2.5° or 1.6°.

\section{Absolute and relative coordinate systems}
The first thing we need to define is the coordinate system to represent where the robot is positioned in the world, this coordinate system will also server to represent where every object in the world is located. This is known as the world coordinate system $\Sigma_W$, its origin can be fixed anywhere in the world.

Where things are located with respect to the robot. 

\section{}

\end{document}

