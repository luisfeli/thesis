\documentclass[a4paper]{report}

\begin{document}

\section{ZMP and Ground Reaction Forces}
To guarantee a walk is stable humanoid robots have to move making sure the sole
of their feet is in contact with the ground all the time. However making contact
is not enough additional constrains have to be met to actually

The Zero Moment Point (ZMP) defined by Vukobratovi\'{c} and Stepanenko <reference>
is one of the key concepts that help to understand the dynamics of walking.
In addition to this concept it is also important to understand its interaction
with the Support Polygon and the ground projection of the center of mass (CoM).

Geometrically the Support Polygon is the region formed by enclosing all contact
points between the robot's feet and the ground, as shown in figure <reference>.
Mathematically the Support Polygon is defined as convex hull <reference>, which
is the smallest set containing all the contact points.

In the following sections a One of the most remarks
The ZMP always exists inside of the support polygon <reference>.

The ground projection of the center of mass is the point where the gravity line
from the center of mass intersects the ground. The projection of the CoM may lie
inside or outside the support polygon. This is shown in figure <reference>.

When a person is standing on the ground, the ZMP coincides with the ground projection
of the CoM. However the projection of the CoM of a person who is practicing a
sport may lie outside the support polygon without compromising his stability but
even in this case the ZMP is inside the support polygon, see figure <reference>.

\subsection{ZMP Analysis}
The free body diagram in figure <reference> shows the friction and reaction forces
applied to the sole of the robot's feet by the ground, these forces are key for
the ZMP analysis.



\end{document}
